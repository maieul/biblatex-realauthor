\documentclass{ltxdockit}[2011/03/25]
\usepackage{btxdockit}
\usepackage{fontspec}
\usepackage[mono=false]{libertine}
\usepackage{microtype}
\usepackage[american]{babel}
\usepackage[strict]{csquotes}
\setmonofont[Scale=MatchLowercase]{DejaVu Sans Mono}
\usepackage{shortvrb}
\usepackage{pifont}
\usepackage{minted}
% Usefull commands
\newcommand{\biblatex}{biblatex\xspace}
\pretocmd{\bibfield}{\sloppy}{}{}
\pretocmd{\bibtype}{\sloppy}{}{}
\newcommand{\namebibstyle}[1]{\texttt{#1}}
% Meta-datas
\titlepage{%
	title={Description of real authors with biblatex},
	subtitle={New data field},
	email={maieul <at> maieul <dot> net},
	author={Maïeul Rouquette},
	revision={2.2.0},
	date={26/04/2015},
	url={https://github.com/maieul/biblatex-realauthor}}

% biblatex
\usepackage[bibstyle=realauthor,citestyle=verbose]{biblatex}
\addbibresource{example.bib}

\begin{document}

\printtitlepage
\tableofcontents
\section{Introduction}

The standard biblatex fields allow to describe the author of a work, with the so called \bibfield{authors} field. However, some works are published without name, or with pseudonyme, but the scholars know the real author. This package adds a new field \bibfield{realauthor}, to specify the knew real author.

\section{Basic use}

\subsection{The .bib file}
Basically, you just have to add the real author name in the field \bibfield{realauthor}, like in the two following examples: 

\inputminted{tex}{example.bib}

\subsection{Loading of the \biblatex package}

When loading the \biblatex package, use the \namebibstyle{realauthor}  bibstyle, which is based on the \namebibstyle{verbose} bibstyle:

\begin{minted}{latex}
\usepackage[citestyle=verbose,bibstyle=realauthor]{biblatex}
\end{minted}

Not that the package patches \verb+author+ and \verb+author/editor+others/translator+others+ macro. If you redefine them, don't forget the patches\footnote{We suppose that, if you redefine them, you could understand the packages.}.

\subsection{Example of result}

By default, the real author name is printed in brackets, with an equal, between fine nonbreak spaces,  when a pseudonym is also use:
\begin{quotation}
\cite{LeClerc1686}

\cite{Simon1686}
\end{quotation}

\subsection{\bibfield{Labelname} field}\label{field:label} 

This package, since version 2.0.0, declares  the \bibfield{realauthor} name as the \bibfield{labelname}, if a \bibfield{author} name is not defined.

That implies this field is use for:
\begin{itemize}
  \item Sorting.
  \item \emph{ibid} / \emph{op. cit} abreviation.
\end{itemize}

Note that this field is not affected by \verb+useauthor+ option, but it is by \verb+userealauthor+ option \secref{userealauthor}.

Used in a bibliography macro or driver, \cmd{ifuserealauthor}\mprm{true}\mprm{false} expands to \prm{true} if the \opt{userealauthor} option is enabled (either globally or for the current entry), and \prm{false} otherwise.

\subsection{Sorting}\label{sorting}

The default sorting schemes \verb+nty+ is problematic, because it doesn't consider the \bibfield{realauthor} field  as a name. That is why the package provides two sorting scheme:

\begin{itemize}
   \item  \verb|anonymous|, which  but considers the \bibfield{realauthor} field to sort list by authors' name. If an entry has both \bibfield{realauthor} and \bibfield{author}, only the \bibfield{author} field we be used.
  \item  \verb|anonymous+realauthor|, which push the anonymous texts at the begininng of the list (as the \verb+anonymous+ sorting of the \emph{biblatex-anonymous+} package), but considers the \bibfield{realauthor} field to sort list by authors' name (as the \verb+realauthor+ sorting).
\end{itemize} 
  See the example file to have an illustration.
  
\section{Disable realauthor use}\label{userealauthor}

The package, or rather \biblatex, defines an \verb+userealauthor+ option, which can be settable on a per-entry basis in the \bibfield{options} field, or by entrytype, using \cmd{ExecuteBibliographyOptions}, or globally, when loading \biblatex. 

This option :
\begin{itemize}
	\item Disable use of \bibfield{realauthor} as label name.
	\item Disable use of \bibfield{realauthor} for sorting.
	\item With the default style, and contrary to the \verb+useauthor+ option, also disable printing of \bibfield{realauthor}.
\end{itemize}


\section{Customization}

You can redefine the \cmd{mkrealauthor} command to change the way the real author name is printed. Default value is:

\begin{minted}{tex}
\newcommand{\mkrealauthor}[1]{\mkbibbrackets{#1}}
\end{minted}

You can also redefine the \cmd{realauthorequalsign} command to change the sign before the real author name when a pseudonym is also used. Default value is:

\begin{minted}{tex}
\newcommand{\realauthorequalsign}{\addnbthinspace=\addnbthinspace}
\end{minted}

For more customization, you have to redefine the macros defined in \file{realauthor.bbx}.




\section{Use with biblatex-manuscripts-philology}

To use this package with the features of \verb+biblatex-manuscripts-philology+, you must use the package \verb+biblatex-multiple-dm+.

\section{Credits}

This package was created for Maïeul Rouquette's phd dissertation\footnote{\url{http://apocryphes.hypothese.org}.} in 2014. It is licensed on the \emph{\LaTeX\ Project Public License}\footnote{\url{http://latex-project.org/lppl/lppl-1-3c.html}.}.


All issues can be submitted, in French or English, in the GitHub issues page\footnote{\url{https://github.com/maieul/biblatex-realauthor/issues}.}.

\section{Change history}

\begin{changelog}

  \begin{release}{2.2.0}{2015-04-26}
  \item Add new \verb|userealauthor| option. Require \biblatex~3.0. \see{userealauthor}
  \item Provide good file version in the \cs{ProvidesFile} command.
  \end{release}

  \begin{release}{2.1.0}{2014-10-30}
  \item Add new \verb|realauthor| scheme. \see{sorting}
  \end{release}


  \begin{release}{2.0.0}{2014-10-27}
  \item Print the realauthor name in bracket when cited in a abreviated form.
  \item Define \bibfield{realauthor} as \bibfield{labelname}. \see{field:label}
  \end{release}

\begin{release}{1.1.0}{2014-10-09}
\item In a bibliographical list, print the last name before the first name of a real author, except if a \bibfield{author} is defined.
\item Add \verb|anonymous+realauthor| sorting scheme.   \see{sorting}
\item Compatibility with \bibtype{article} entry. 
\end{release}

\begin{release}{1.0.0}{2014-06-25}
\item First public release.
\end{release}
\end{changelog}
\end{document}
