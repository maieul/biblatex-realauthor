\documentclass{ltxdockit}[2011/03/25]
\usepackage{btxdockit}
\usepackage{fontspec}
\usepackage[mono=false]{libertine}
\usepackage{microtype}
\usepackage[american]{babel}
\usepackage[strict]{csquotes}
\setmonofont[Scale=MatchLowercase]{DejaVu Sans Mono}
\usepackage{shortvrb}
\usepackage{pifont}
\usepackage{minted}
% Usefull commands
\newcommand{\biblatex}{biblatex\xspace}
\pretocmd{\bibfield}{\sloppy}{}{}
\pretocmd{\bibtype}{\sloppy}{}{}
\newcommand{\namebibstyle}[1]{\texttt{#1}}
% Meta-datas
\titlepage{%
	title={Description of real authors with biblatex},
	subtitle={New data field},
	email={maieul <at> maieul <dot> net},
	author={Maïeul Rouquette},
	revision={1.0.0},
	date={15/04/2014},
	url={https://github.com/maieul/biblatex-realauthor}}

% biblatex
\usepackage[bibstyle=realauthor]{biblatex}
\addbibresource{example.bib}

\begin{document}

\printtitlepage
\tableofcontents
\section{Introduction}

The standard biblatex fields allow to describe the author of a work, with the so called \bibfield{authors} field. However, some works are published without name, or with pseudonyme, but the scholars know the real author. This package adds a new field \bibfield{realauthor}, to specify the knew real author.

\section{Basic use}

\subsection{The .bib file}
Basically, you just have to add the real author name in the field \bibfield{realauthor}, like in the two following examples: 

\inputminted{tex}{example.bib}

\subsection{Loading of the \biblatex package}

When loading the \biblatex package, use the \namebibstyle{realauthor}  bibstyle, which is based on the \namebibstyle{verbose} bibstyle:

\begin{minted}{latex}
\usepackage[citestyle=verbose,bibstyle=realauthor]{biblatex}
\end{minted}

\subsection{Example of result}

By default, the real author name is printed in brackets, with an equal, between fine nonbreak spaces,  when a pseudonym is also use:
\begin{quotation}
\cite{LeClerc1686}

\cite{Simon1686}
\end{quotation}

\section{Change history}

\begin{changelog}



\begin{release}{1.0.0}{2014-04-15}
\item First public release.
\end{release}
\end{changelog}
\end{document}
